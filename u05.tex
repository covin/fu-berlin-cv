\documentclass[a4paper,10pt]{report}

\topmargin -2cm
%\topskip0cm
%\footskip0cm
%\headsep0cm
\parindent0cm
\oddsidemargin -1cm
\evensidemargin -1cm
\headheight 2cm
\textheight 24cm
\textwidth 18cm

\author{Alexander M\"unn (4403061)}
\title{\"Ubung}

\include{tex/header}
\usepackage{fancyhdr}
\pagestyle{fancy}
\lhead{Michael Borst\\ Alex Muenn}
\chead{"Ubungsblatt \nr\\\today}
\rhead{Computer Vision}



\newcommand{\nr}{5}

\begin{document}
\section*{Aufgabe 10 - Kamera-Kalibrierung mit vier Punkten in einer Ebene}


\lstset{language=matlab}
\begin{lstlisting}[caption={Funktion f\"ur die Kamera-Kalibrierung}]
% returns calibration with rotation R and translation t
function [R t] = locateCam(I, W)
  [points, dim] = size(I);
  if points != 4
    printf('are you kidding me? I exactly need four points\n');
  endif
  if size(I) != size(W)
    printf('invalid coordinate mapping\n');
  endif
  x = 1;
  y = 2;

  F = [];
  for p=1:points
     F = [F; W(p,x) W(p,y) 1      0      0 0 -I(p,x)*W(p,x) -I(p,x)*W(p,y) -I(p,x)];
     F = [F;      0      0 0 W(p,x) W(p,y) 1 -I(p,y)*W(p,x) -I(p,y)*W(p,y) -I(p,y)];
  end
  h = inv(F(:,1:8))*-F(:,9);
  H = [h(1) h(2) h(3)
       h(4) h(5) h(6)
       h(7) h(8)   1];
  H /= norm(H(:,1));
  r1 = H(:,1);
  r2 = H(:,2);
  r3 = cross(r1, r2);
  R  = [r1 r2 r3];
  t  = -R^-1*H(:,3);
end
\end{lstlisting}

\begin{lstlisting}[caption={Anwendung der Kalibrierung}]
% Manuelle Bestimmung der Punkte
% P5100207s: P1(400|160), P2(463|343), P3(166|391), P4(118|190)
Pic1 = [400, 160; 463, 343; 166, 391; 118, 190];
% P5100208s: P1(427|143), P2(567|300), P3(185|376), P4( 88|222)
Pic2 = [427, 173; 567, 300; 185, 376;  88, 222];
% P5100209s: P1(426|166), P2(589|269), P3(177|344), P4( 65|215)
Pic3 = [426, 166; 589, 269; 177, 344;  65, 215];
% P5100210s: P1(435|166), P2(611|231), P3(185|279), P4( 63|196)
Pic4 = [435, 166; 611, 231; 185, 279;  69, 196];
World = [ 10,  10
          10, 220
         317, 220
         317,  10];
% sensor dimension of [17.3mm x 13.0mm]
dim = [640, 480];
sen = [17.3, 13];
f   = 38;

function S = pixel2sensor(P, Dim, Sensor)
    S = (P / diag(Dim)-0.5) * diag(Sensor);
end
format = 'Cam position in pic %d: x=%.2f cm,y=%.2f cm, h=%.2f cm\n'; 
S = pixel2sensor(Pic1, dim, sen)./f;
[R t] = locateCam(S,World);
printf( format, 1, t(1)/10, t(2)/10, t(3)/10);
% Orientierung der Kamera entspricht der Z-Achse
O = (R*[0 0 1]')'

S = pixel2sensor(Pic2, dim, sen)./f;
[R t] = locateCam(S,World);
printf(format,  2, t(1)/10, t(2)/10, t(3)/10);
O = (R*[0 0 1]')'

S = pixel2sensor(Pic3, dim, sen)./f;
[R t] = locateCam(S,World);
printf(format,  2, t(1)/10, t(2)/10, t(3)/10);
O = (R*[0 0 1]')'

S = pixel2sensor(Pic4, dim, sen)./f;
[R t] = locateCam(S,World);
printf(format,  2, t(1)/10, t(2)/10, t(3)/10);
O = (R*[0 0 1]')'
\end{lstlisting}

\lstset{language=matlab}
\begin{lstlisting}[caption={Ausgabe}]
octave:14> u05
Cam position in pic 1: x=63.24 cm,y=67.04 cm, h=113.33 cm
O =

   0.27315  -0.51093  -0.85977

Cam position in pic 2: x=60.00 cm,y=89.33 cm, h=50.62 cm
O =

   0.091799  -0.914931  -0.504779

Cam position in pic 2: x=60.01 cm,y=89.53 cm, h=37.79 cm
O =

   0.045332  -0.966189  -0.414578

Cam position in pic 2: x=60.13 cm,y=89.00 cm, h=21.07 cm
O =

   0.024955  -1.057201  -0.277447
\end{lstlisting}

\end{document}
